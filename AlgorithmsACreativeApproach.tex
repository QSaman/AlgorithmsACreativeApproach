\documentclass{book} 

\usepackage{graphicx}
\usepackage{amsmath}
\usepackage{algorithm}
\usepackage{algpseudocode}
\usepackage{float}
\usepackage{hyperref}
\usepackage{placeins}
\usepackage{amssymb}

\usepackage{tikz}
\usetikzlibrary{arrows}
\usetikzlibrary{snakes}
\usetikzlibrary{decorations.pathmorphing}

\hypersetup{
	colorlinks,
	citecolor=black,
	filecolor=black,
	linkcolor=black,
	urlcolor=black
}

\title{Solutions of Introduction to Algorithms: A Creative Approach}
\author{Saman Saadi}
\date{} 

\begin{document}
	\frontmatter
	\maketitle
%	\newpage
	\tableofcontents
	\mainmatter
	\chapter{Mathematical Induction}
	\section{Counting Regions in the Plane}
	A set of lines in the plane is said to be in \textbf{general position} if no two lines are parallel and no three lines intersect at a common point.
	\paragraph{Guess:} Adding one more line to $n - 1$ lines in general position in the plane \textbf{increases} the number of regions by $n$. In other words $T(n) = T(n - 1) + n$.
	\par The base cases is trivial
	\begin{itemize}
		\item $T(0) = 1$
		\item $T(1) = T(0) + 1 = 2$
		\item $T(2) = T(1) + 2 = 2 + 2 = 4$
		\item $T(3) = T(2) + 3 = 4 + 3 = 7$
	\end{itemize}
	So we assume $T(n)$ is correct, now we want to prove $T(n + 1)$ is also correct. Let's remove line $n^{th}$. According to induction hypothesis Adding line $(n+1)^{th}$ add $n$ new regions. If we add line $n^{th}$ again, it intersect with line $(n+1)^{th}$ at exactly one point $p$. This point is located in region $R$.
	\par In the absence of line $n^{th}$, line $(n+1)^{th}$ adds only one new region when it passes $R$. But in presence of line $n^{th}$, it adds 2 new regions when it passes $R$. For other regions line $(n+1)^{th}$ adds $n - 1$ new regions with or without the presence of line $n^{th}$. So line $(n+1)^{th}$ adds $n - 1 + 2 = n + 1$ new regions when $n^{th}$ is presented.
\end{document}
